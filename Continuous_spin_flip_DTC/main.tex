\documentclass[a4paper, 11pt]{article}
\usepackage[left=1in, right =1in, top=1in, bottom=1in]{geometry}
% Useful packages
\usepackage{amsmath,physics, amssymb}
\usepackage{graphicx}
\usepackage[colorlinks=true, allcolors=blue]{hyperref}
\usepackage{color}
\newcommand{\blue}[1]{\textcolor{blue}{#1}}
\newcommand{\red}[1]{\textcolor{red}{#1}}
\usepackage{soul, ulem, todonotes}
%\usepackage{mathptmx}

\usepackage[capitalise]{cleveref}
%\usepackage[backend=bibtex, citestyle=numeric-comp, sorting=none, url =false]{biblatex}
\usepackage[backend=biber, style=numeric]{biblatex}
\addbibresource[]{main.bib}  
% Include bibliography from discussion.bib
% The bibliography will be printed at the end using \printbibliography{}
\setcounter{tocdepth}{2}

% Sans modern font
%\usepackage{sansmath}
%\renewcommand{\familydefault}{\sfdefault}
%\sansmath

\title{Discrete time crystal in continuous spin flip protocol}
\usepackage{authblk}
\author[1]{Mahbub Rahaman\thanks{\texttt{mahabubrahaman@hri.res.in}}}
\author[1]{Sayan Choudhury\thanks{\texttt{sayanchoudhury@hri.res.in}}}
\affil[1]{\small Harish-Chandra Research Institute, HBNI, Chhatnag Road, Jhunsi, Praygraj, UP - 211019, India}
\date{}
\begin{document}
\maketitle
%\tableofcontents
%\newpage

\section{The Model and Dynamics}
We consider a one-dimensional chain of spin-$1/2$ particles described by a time-dependent Hamiltonian comprising two components:
\begin{align}
    \hat{\mathcal{H}}_{\text{total}}(t) &=  \hat{\mathcal{H}}_0(t) + \hat{\mathcal{H}}_1(t), \\
    \hat{\mathcal{H}}_0(t) &= J\sum_{j} \hat{\sigma}_j^z \hat{\sigma}_{j+1}^z, \\
    \hat{\mathcal{H}}_1(t) &= G(t)\sum_{j}\hat{\sigma}_j^x, \quad G(t) = g_0\sin^2\left(\frac{\omega}{2} t\right)
\end{align}
Accordingly, the total Hamiltonian is expressed as
\begin{equation}
    \boxed{
        \hat{\mathcal{H}}_{\text{total}}(t) =  J\sum_{j} \hat{\sigma}_j^z \hat{\sigma}_{j+1}^z + g_0\sin^2\left(\frac{\omega}{2} t\right)\sum_{j}\hat{\sigma}_j^x
    }
\end{equation}
where $J$ denotes the nearest-neighbor Ising interaction strength, $\hat{\sigma}_j^{x,z}$ are the Pauli matrices acting on the $j$-th spin, and $G(t)$ represents a time-dependent transverse field characterized by amplitude $g_0$ and frequency $\omega$. The amplitude $g_0$ is chosen such that $\displaystyle \int_0^T G(t)\,dt = \frac{\pi}{2}$, ensuring that the transverse field induces a spin flip over one period $T = \frac{2\pi}{\omega}$. Consequently, the system is anticipated to exhibit a subharmonic response with a period of $2T$.

Explicitly,
\begin{align*}
    &\int_0^T G(t)\,dt = \frac{\pi}{2} \\
    &\int_0^T g_0\sin^2\left(\frac{\omega}{2}t \right) dt = \frac{\pi}{2}\\
    &g_0 \frac{\pi}{\omega} = \frac{\pi}{2}\\
    &\boxed{g_0 = \frac{\omega}{2}}
\end{align*}

The system is initialized in a fully polarized state, with all spins aligned in the up direction: $\displaystyle\ket{\psi(0)} = \ket{\uparrow\uparrow\uparrow\ldots\uparrow}$. The time evolution is governed by the time-ordered exponential of the total Hamiltonian:
\begin{equation}
    \ket{\psi(t)} = \mathcal{T} \exp\left(-i\int_0^t \hat{\mathcal{H}}_{\text{total}}(t')\,dt'\right) \ket{\psi(0)}.
\end{equation}

\end{document}